\documentclass[12pt]{article}

\usepackage{geometry}

\usepackage{amsmath}
\usepackage{amsfonts}
\usepackage{graphicx}
\usepackage[T1]{fontenc}
\usepackage[french]{babel}
\usepackage[autolanguage]{numprint} % for the \nombre command
\usepackage{hyphenat}
%\hyphenation{mate-mática recu-perar}
\usepackage{amsthm}
\usepackage{enumitem}
\usepackage[hidelinks]{hyperref}
\hypersetup{
	colorlinks,
	citecolor=black,
	filecolor=black,
	linkcolor=black,
	urlcolor=black
}
\usepackage{float}


\usepackage[dvipsnames]{xcolor}

\newcommand{\defi}{\paragraph{Définition.}}
\newcommand{\powerset}{\mathcal{P}}


\theoremstyle{plain}
%\newtheorem{thm}{Théorème}[chapter] % reset theorem numbering for each chapter
%\newtheorem{lm}{Lemme}[thm]
%\newtheorem{cor}{Corollaire}[thm]

\theoremstyle{definition}
%\newtheorem{defn}[thm]{Définition} % definition numbers are dependent on theorem numbers
%\newtheorem{ex}[thm]{Example} % same for example numbers
%\newtheorem{rmrk}[thm]{Remarque}

\graphicspath{ {./images/} }


% Paramètres de mise en page
\geometry{a4paper, margin=0.75in}



\setcounter{section}{1}
%\renewcommand*{\thesection}{\Roman{section}.}
\renewcommand*{\thesection}{\arabic{section}.}
%\renewcommand*{\thesubsection}{\arabic{subsection}.}
\renewcommand*{\thesubsection}{\alph{subsection}.}
\renewcommand*{\thesubsubsection}{\roman{subsubsection}.}


\begin{document}
\begin{titlepage}
	
    \centering
    {\color{red}\hrule width \hsize height 1.5mm \kern 1mm  \color{Goldenrod}\hrule width \hsize height 0.5mm}
    %{\color{red} \rule{\linewidth}{1.5mm}}
    %{\color{Goldenrod} \rule{\linewidth}{0.25mm}}
    \vspace*{4cm}
    \includegraphics[width=0.5\textwidth]{ulaval.jpg}
    \vspace*{2cm}

    {\LARGE\textbf{MAT-2400}}\\[0.5cm]
    {\Large\textbf{Méthodes numériques }}\\[2cm]

    {\large\textbf{Devoir 3}}\\[0.5cm]
    {\large\textbf{\today}}\\[2cm]

    {\large\textbf{Louis-Victor Carrier-Favreau}}\\[0.3cm]
    {\large\textbf{NI. 537 320 649}}\\[0.6cm]
    {\large\textbf{Jean-Christophe Parent}}\\[0.3cm]
    {\large\textbf{NI. ??? ??? ???}}\\[3cm]

    \vfill
    {\color{Goldenrod}\hrule width \hsize height 0.5mm \kern 1mm \color{red}\hrule width \hsize height 1.5mm }
    %{\color{Goldenrod} \rule{\linewidth}{0.25mm}}
    %{\color{red} \rule{\linewidth}{1.5mm}}
\end{titlepage}


\tableofcontents
\listoffigures
%\listoftables
\newpage

\begin{figure}[H]
	\centering
	\includegraphics[width=\textwidth]{fig1.png}
	\caption{Polynome d'interpolation de \(f(x)\)}
\end{figure}

On voit que le polynome d'interpolation ne converge pas vers \(f(x)\) lorsque $n\to\infty$. En fait, comme l'interpolation est un polynome de degré \(n\) on voit que si \(P_n(x)\to f(x)\) \((n\to\infty)\) alors \(P_n\to T_0(x)\) \((n\to\infty)\) où \(T_0(x)\) est le polynome de Taylor de \(f\) en \(0\). Cependant \(f\) est discontinue en \(x=\pm i\) comme Mathieu l'a mentionné en classe. Donc par le théorème de Taylor le rayon de convergence est \(R=|i|\). Comme l'interpolation est sur \([-5,5]\)  qui est en dehors de $D_1(0)\subset\mathbb{C}$ l'interpolation ne peux pas converger vers $T_0(x)$.

\begin{figure}[H]
	\centering
	\includegraphics[width=\textwidth]{fig2.png}
	\caption{Maximum de l'erreur du polynome d'interpolation de \(f(x)\) en fonction de \(h\)}
\end{figure}

\begin{figure}[H]
	\centering
	\includegraphics[width=\textwidth]{fig3.png}
	\caption{Interpolation quadratique par morceaux \(Q_n(x)\) de \(f(x)\)}
\end{figure}

\begin{figure}[H]
	\centering
	\includegraphics[width=\textwidth]{fig4.png}
	\caption{Erreur max de \(Q_n(x)\) en fonction de \(h\)}
\end{figure}

La figure montre que $E(h)$ diminue proportionellement à $h^3$. On observe le comportement attendus entre l'erreur et la courbe de référence ce qui confirme que l'interpolation quadratique par morceaux converge avec un ordre de $3$, comme démontré ci-bas.

\begin{figure}[H]
	\centering
	\includegraphics[width=\textwidth]{fig5.png}
	\caption{Splines cubiques \(S_n(x)\) de \(f(x)\)}
\end{figure}

On voit sur la figure le résultat est lisse contrairement aux $Q_n$ et que le résultat est plus proche de la fonction originale.

\section{Démonstration que \(E(h)\leq Ch^3\)}
\begin{align*}
f(x)&=\frac{1}{1+x^2}\\
|f'''(x)|&\leq M\ (M>0)\\
E(h)&:=\max_{-5\leq x\leq5}|f(x)-Q_n(x)|
\end{align*}

Travaillons sur \(I=[x_i,x_{i+2}]\subset[-5,5]\). Soit \(\xi_x\in I\).
\begin{align*}
\left|f(x)-Q_n(x)\right|&=\left|\frac{f'''(\xi_x)}{3!}(x-x_i)(x-x_{i+1})(x-x_{i+2})\right|\\
&=\left|\frac{f'''(\xi_x)}{3!}(x-x_i)(x-(x_i+h))(x-(x_i+2h))\right|\\
&=\left|\frac{f'''(\xi_x)}{3!}(x-x_i)((x-x_i)-h)((x-x_i)-2h)\right|\\
&\leq\left|\frac{M}{3!}(x-x_i)((x-x_i)-h)((x-x_i)-2h)\right|\\
\end{align*}

Comme \(|x-x_i|\leq 2h\) on trouve que \(|x-x_i-h|\leq h\) et \(|x-x_i-2h\leq 2h\) et donc

\begin{align*}
\left|f(x)-Q_n(x)\right|&\leq\frac{M}{3!}2h\times h\times2h\\
&=\frac{2M}{3}h^3\\
&=Ch^3\ \left(C=\frac{2M}{3}\right)
\end{align*}

Par conséquent 
\[E(h)=\max_{-5\leq x\leq5} \left|f(x)-Q_n(x)\right| \leq Ch^3\]


\end{document}
